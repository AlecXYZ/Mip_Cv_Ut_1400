\documentclass[10pt,twoside,slovak,a4paper]{article}

\usepackage[slovak]{babel}
\usepackage[IL2]{fontenc}
\usepackage[utf8]{inputenc}
\usepackage{graphicx}
\usepackage{url}
\usepackage{hyperref}
\usepackage{cite}

\pagestyle{headings}

\title{Hry ako jeden z nástrojov vyučovacieho procesu\thanks{Semestrálny projekt v predmete Metódy inžinierskej práce, ak. rok 2022/23, vedenie: Meno Priezvisko}}

\author{Meno Priezvisko\\[2pt]
	{\small Slovenská technická univerzita v Bratislave}\\
	{\small Fakulta informatiky a informačných technológií}\\
	{\small \texttt{...@stuba.sk}}
	}

\date{\small 6. november 2022}



\begin{document}

\maketitle

\begin{abstract}

Moderné technológie, ktoré sú súčasťou nášho každodenného života, ponúkajú široké možnosti využitia v rôznych odboroch a sférach. Jednou z nich je aj použitie hier v procese výučby\cite{Zea2009-eh}, v ktorom v súčasnosti nezohrávajú veľkú rolu, napriek svojmu potenciálu. Vo svojom článku sa budem venovať práve tomuto potenciálu, teda možnostiam využitia hier vo vzdelávacom procese, čo by pomohlo zjednodušiť a zlepšiť celkovú výučbu. Zameriam sa na porovnanie výučby klasickej a výučby s použitím vzdelávacích hier. Zároveň poukážem na výhody plynúce z využitia hier v samotnom procese výučby v jednotlivých stupňoch vzdelávania. Pre každý stupeň vyberiem konkrétny príklad, na ktorom vysvetlím prínos herných prvkov do vyučovania a uvediem akým spôsobom rozvíjajú zručnosti a schopnosti žiaka.

\end{abstract}



\section{Úvod}


\section{Výhody použitia hier vo výučbe}

Videohry ako také sa v dnešnej dobe stali novou voľnočasovou aktivitou detí. Už od najmenšieho veku sa deti oboznamujú s videohrami, kde si môžu rozvíjať motoriku, reflexy, trpezlivosť, vytrvalosť a ďalšie schopnosti. Digitálne hry, ak sú kvalitne vytvorené, dokážu upútať pozornosť študentov viac, ako klasické vyučovacie prostriedky.

\subsection{Prínosy pre poslucháčov}

\subsection{Prínosy pre vyučúcich}

\section{Nevýhody použitia hier vo výučbe}

\section{Iná časť}



Niekedy treba uviesť zoznam:

\begin{itemize}
\item jedna vec
\item druhá vec
	\begin{itemize}
	\item x
	\item y
	\end{itemize}
\end{itemize}

Ten istý zoznam, len číslovaný:

\begin{enumerate}
\item jedna vec
\item druhá vec
	\begin{enumerate}
	\item x
	\item y
	\end{enumerate}
\end{enumerate}


\subsection{Ešte nejaké vysvetlenie} \label{ina:este}

\paragraph{Veľmi dôležitá poznámka.}
Niekedy je potrebné nadpisom označiť odsek. Text pokračuje hneď za nadpisom.



\section{Dôležitá časť} \label{dolezita}




\section{Ešte dôležitejšia časť} \label{dolezitejsia}




\section{Záver} \label{zaver} % prípadne iný variant názvu



\bibliography{literatura}
\bibliographystyle{plain}
\end{document}
