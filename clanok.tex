\documentclass[10pt,twoside,slovak,a4paper]{article}

\usepackage[slovak]{babel}
\usepackage[IL2]{fontenc}
\usepackage[utf8]{inputenc}
\usepackage{graphicx}
\usepackage{url}
\usepackage{hyperref}
\usepackage{cite}

\pagestyle{headings}

\title{Hry ako jeden z nástrojov vyučovacieho procesu\thanks{Semestrálny projekt v predmete Metódy inžinierskej práce, ak. rok 2022/23, vedenie: Meno Priezvisko}}

\author{Meno Priezvisko\\[2pt]
	{\small Slovenská technická univerzita v Bratislave}\\
	{\small Fakulta informatiky a informačných technológií}\\
	{\small \texttt{...@stuba.sk}}
	}

\date{\small 14. december 2022}



\begin{document}

\maketitle

\begin{abstract}

Moderné technológie, ktoré sú súčasťou nášho každodenného života, ponúkajú široké možnosti využitia v rôznych odboroch a sférach. Jednou z nich je aj použitie hier v procese výučby, v ktorom v súčasnosti nezohrávajú veľkú rolu, napriek svojmu potenciálu. Vo našom článku sa budeme venovať práve tomuto potenciálu, teda možnostiam využitia hier vo vzdelávacom procese, čo by pomohlo zjednodušiť a zlepšiť celkovú výučbu. Zameriame sa na porovnanie výučby klasickej a výučby s použitím vzdelávacích hier. Zároveň poukážeme na výhody plynúce z využitia hier v samotnom procese výučby v jednotlivých stupňoch vzdelávania. Pre každý stupeň vyberieem konkrétny príklad, na ktorom vysvetlíme prínos herných prvkov do vyučovania a uvedieme akým spôsobom rozvíjajú zručnosti a schopnosti študenta.

\end{abstract}



\section{Úvod}

Dnešná doba je plná technológií, ktoré nám poskytujú množstvo možností vo všetkých sférach nášho každodenného života. Výnimkou nie je ani proces vyučovania, v ktorom nám moderné technológie, konkrétne hry\cite{Zea2009-eh}, môžu pomôcť. Tie nám ponúkajú veľké množstvo príležitostí, ktoré priam vyzývajú k tomu, aby sa využívali a doplnili tak alebo úplne nahradili klasické vyučovacie metódy. Najmä v posledných rokoch začínajú vychádzať na povrch nedostatky klasického vyučovania. Svojou monotónnosťou, pasivitou a zložitosťou pôsobia na študentov demotivujúco a pomaly, ale isto im uberajú z chuti vzdelávať sa. Tieto nedostatky by mohlo napomôcť vyriešiť práve využitie hier v procese vyučovania, ktorému sa budeme venovať v našom článku. 

Cieľom výučby prostredníctvom videohier je čo najviac zaujať študentov, rozvíjať a podporovať v nich logické a tvorivé myslenie a viesť ich k samostatnej práci\cite{Chen2012-ao}. Tento článok sa snaží poukázať na výhody (kap.~\ref{vyhody}) a nevýhody (kap.~\ref{nevyhody}) využitia videohier v samotnom procese výučby a zároveň analyzoje spôsoby použitia tejto metódy v jednotlivých stupňoch vyučovania.

\section{Výhody použitia hier vo výučbe} \label{vyhody}

Videohry ako také sa v dnešnej dobe stali novou voľnočasovou aktivitou detí. Už od najmenšieho veku sa deti oboznamujú s videohrami, kde si môžu rozvíjať motoriku, reflexy, trpezlivosť, vytrvalosť a ďalšie schopnosti. Digitálne hry, ak sú kvalitne vytvorené, dokážu upútať pozornosť študentov viac, ako klasické vyučovacie prostriedky. 

V súčasnosti už prebehli a zároveň prebiehajú rôzne výskumy, ktoré skúmajú čo prinášajú tieto metódy priamo v aktívnom procese výučby\cite{Chen2012-ao}. Je známe, že výučba hrou je omnoho prijateľnejšia pre väčšinu študentov ako výučba klasická. Aj pri menej obľúbenom predmete má vyučujúci prostredníctvom hry viac možností zatraktívniť tento predmet a zaujať tak viac poslucháčov, urobiť z tohto predmetu zábavu a nenásilným spôsobom priviesť študentov k vedomostiam bez toho, že by si to primárne uvedomovali. Tento spôsob výučby má mnoho kladov ako pre poslucháčov (podkap.~\ref{vyhody:studenti}), tak i pre vyučujúcich (podkap.~\ref{vyhody:vyucujuci}).

\subsection{Prínosy pre študentov} \label{vyhody:studenti}

Zásadnou výhodou takejto metódy výučby je možnosť aktívneho zapojenia študentov do vyučovacieho procesu. Je všeobecne známe, že ak si študent preberanú látku prakticky vyskúša, upevní si tak získané vedomosti. Avšak pri klasickej metóde vyúčby bolo takmer jedinou možnosťou aktívneho zapojenia sa do vyučovania skúšanie pri tabuli, ktoré je pre mnohých študentov stresujúce. Využitím hier by sa tento problém vyriešil, pretože pri samotnom hraní by si študent upevnil alebo doplnil svoje vedomosti.

 Ďalšou motiváciou je ocenenie úspechu v rámci procesu vyučovania. Zároveň prípadný neúspech nepôsobí demotivujúco a to najmä z dôvodu možnosti opakovania nesprávne vyriešenej časti hry. Týmto spôsobom je študentovi umožné lepšie analyzovať chyby, ktorých sa pri riešení dopustil, čo by napríklad pri riešení pracovného listu nebolo možné. Študentov bude taktiež motivovať už len samotný fakt, že namiesto niekoľkých hodín, ktoré by museli stráviť nad učebnicami, budú získavať nové znalosti prostredníctvom videohry\cite{7930309}. 

Nesmieme zabudnúť ani na výhodu osvojenia si cudzieho jazyka, pretože v mnohých hrách sa nevyhneme ich použitiu. A v neposlednom rade, pri hraní prebieha podvedomý rozvoj schopností ako tvorivosť, seba vyjadrenie, preberanie zodpovednosti za svoje rozhodnutia\cite{6185607}, prispôsonie sa, stanovovanie si primeraných cielov a mnohé ďalšie.

\subsection{Prínosy pre vyučúcich} \label{vyhody:vyucujuci}

Využitie hier vo vyučovaní zväčšuje záujem o predmet zo strany študentov\cite{Ucenie}, čo učiteľovi zľahčuje celý vyučovací proces, pretože nemusí venovať toľko usília, aby zaujal študentov a môže sa tak naplno venovať preberanej látke. Za prínos, ktorý zľahčuje prácu vyučujúcemu, môžeme považovať aj automatickú kontrolu a vyhodnotenie výsledkov prejdenej časti hry. S tým súvisí aj ďalšia výhoda, ktorou je rýchla spätná väzba. Tú učiteľ obdrží hneď po ukončení segmentu hry a tak môže prakticky okamžite začať s jej analýzou. Takto môže veľmi rýchlo spoznať silné a slabé stránky študentov, na základe ktorých môže následne prispôsobiť metódy výučby a tak efektívnejšie dosahovať stanovené ciele vzdelávania.

\section{Nevýhody použitia hier vo výučbe} \label{nevyhody}

Pri zavádzaní hier do systému výučby sa však stretávame aj z mnohými nevýhodami.Najčastejšie je problém v technickom riešení na školách, chýbajú moderné a výkonné počítače, ktoré by spĺňali požadované parametre\cite{Videohry}. Pre vyučujúcich sú dané určité oklieštené osnovy pre vzdelávací proces, čo vytvára veľmi malý priestor na zavádzanie nových metód. Medzi nevýhody môžeme zaradiť aj nevôľu zo strany niektorých vyučujúcich zavádzať nové systémy v rámci výučby. Môže to byť spôsobené nezáujmom vyučujúceho o hry ako také alebo je problémom vek a s ním spojené vnímanie zmien a prístup k moderným technológiám. 

Zavedenie videohier do procesu výučby je náročné aj z časového hľadiska, pretože na to, aby videohru mohol vyučujúci priniesť a použiť počas vyučovania, sa s ňou musí najskôr zoznámiť sám. V prvom rade, to znamená vyhľadať vhodnú hru na daný predmet (kap.~\ref{kriteria}), naučiť sa ju ovládať, vyhľadať v nej konkrétne možnosti, ktoré sa dajú použiť v danom predmete a potom musí hru zapracovať do učebných osnov.

\section{Kritéria pri výbere hier} \label{kriteria}

Pri výbere vhodných hier k výučbe je treba, aby vyučujúci prihliadal na určité kritéria, ktoré by mali vybrané hry spĺňať. V prvom rade je dôležité, aby hra študentom poskytovala hodnotné informácie, ktoré im uľahčia pochopiť obsah daného učiva. Umožnila im rozširovať si vedomosti a obzory v rámci vyučovacieho procesu. A v neposlednom rade zaviedla do výučby nové a moderné techniky, ktoré by pomáhali rozvíjať schopnosti jednotlivca. Hra by nemala byť časovo náročná, musí byť použiteľná v praxi a mala by byť vhodná pre určenú skupinu študentov na základe úrovne dosiahnutých vedomostí.

\section{Využitie hier vo výučbe} \label{vyuzitie}

V súčasnosti je k dispozícii veľké množstvo hier rôznych žánrov, avšak problémom je, že nie všetky je možné použiť vo vyučovacom procese. Nato, aby sme nejakú hru mohli začleniť do vyučovania, musí spĺňať určité kritéria (kap.~\ref{kriteria}). Samozrejme hier, ktoré by sto percentne spĺňali všetky kritériá je veľmi málo, ak vôbec také jestvujú. Jedinou možnosťou je preto využitie viacerých hier v rámci jedného predmetu, z ktorých by sa každá zameriavala na jednu určitú problematiku. Spolu by tak tvorili jeden celok, v ktorom by sa navzájom dopĺňali a nadväzovali na seba. Avšak nájdenie takejto skupiny hier, taktiež nie je jednoduchou úlohou, preto sa v súčastnosti ponúka len jedno riešenie a to využitie hier len v určitých častiach vyučovanej látky daného predmetu. Takto síce nebude využitý plný potenciál vzdelávania prostredníctvom videohier, ale aj táto malá zmena by bola u mnohých študentov vítaná. 

Niekto by mohol namietať, že na trhu nie je dostatok hier, ktoré by sa dali označiť za výukové. Toto tvrdenie je samozrejme pravdivé, pretože cieľom drvivej väčšiny hier je zabaviť a nie vzdelávať. Avšak nato, aby sa hra mohla začleniť do vyučovacieho procesu, nemusí byť nutne výuková. Stačí ak hoc len malá časť z jej obsahu zodpovedá nejakej časti preberanej látky\cite{Videohry}. 

Zopár príkladov oboch typov hier, teda výukových, ale aj tých, z ktorých je pre vyučovanie vhodná len ich časť, sme uviedli v tabuľke (tab.~\ref{tab:vyuzit-hry}). Okrem ich názvu je v tabuľke uvedený jeden alebo viac predmetov spolu so stupňom vzdelávania, v ktorom by sa dali uplatniť. Nasledujúce podkapitoly budú venované trom hrám, z ktorých by každú bolo možné využiť v jednom alebo viacerých stupňoch vzdelávania.

\begin{table}[tbh]
	\centering
	\caption{Vhodné hry pre výučbu (A. Rybár, 2022)}
	\begin{tabular}{p{1.2in} p{1.7in} p{1.6in}}
		Názov & Predmet & Stupeň vzdelávania\\ \hline
		Leoncio and friends & Anglický jazyk & Materské a základné školy\\
		Spore & Biológia & Základné školy\\
		Age of Empires & Dejepis & Základné a stredné školy\\
		Duolingo & Cudzie jazyky & Základné a stredné školy\\
		Minecraft & Geografia, Dejepis. Biológia & Základné a stredné školy\\
		Civilization & Dejepis & Základné a stredné školy\\
		Human Anatomy & Medicína, Biológia & Vysoké a stredné školy\\
	\label{tab:vyuzit-hry}
	\end{tabular}
\end{table}

\subsection{Leoncio and friends} \label{hra-1}

Jednou z hier využiteľných v základných a materských školách je "Leoncio and friends"\cite{Zea2009-eh}. Jej hlavným cieľom je vzdelávať, konkrétne naučiť žiakov písať, čítať a vyslovovať samohlásky. Celý proces výučby, ktorý prebieha počas hry, je sprevádzaný jednoduchým príbehom, v ktorom zohráva hlavnú úlohu Leoncio. Jeho priatelia boli unesení hlavným záporákom a úlohou žiakov je vyslobodiť ich. Nato, aby sa im to podarilo musia prekonať niekoľko výziev. Tie sú umiestnené na jednotlivých ostrovoch, z ktorých sa skladá mapa hry. Každá z výziev sa zamieriava na jednu samohlásku, z ktorou by sa mali žiaci oboznámiť počas jej plnenia. Okrem výučby samohlások, hra podvedome rozvíja aj ďalšie schopnosti a zručnosti žiakov, ako napríklad týmovú prácu.

\begin{figure}[tbh]
	\centering
	\includegraphics[scale=0.4]{Img-1.png}
	\caption{"Leoncio and friends" (Zdroj:\cite{Zea2009-eh})}
	\label{obr-1}
\end{figure}

\subsection{Civilization} \label{hra-2}

Pre stredné a základné školy je príkladom hry, ktorá by mohla doplniť vyučovací proces aj napriek tomu, že jej hlavným cieľom nie je vzdelávať, "Civilization"\cite{Civil}. Ide o ťahovú stratégiu, ktorá sa zameriava na vývoj civilizácie od obdobia staroveku po súčasnosť. Súčasťou hry je možnosť výberu jedného z viacerých historických scenárov, ktoré sa v minulosti odohrali. Študent sa tak ocitne v roli jedného z vodcov skutočnej historickej civilizácie a môže tak aspoň z malej časti prežiť dejiny "na vlastnej koži".

\begin{figure}[tbh]
	\centering
	\includegraphics[scale=0.3]{Img-2.png}
	\caption{"Civilization" (Zdroj:\cite{Img-Civil})}
	\label{obr-2}
\end{figure}

\subsection{Human Anatomy} \label{hra-3}

Simulátor "Human Anatomy"\cite{Antom} je určený najmä pre vysoké a stredné školy. Študentom prezentuje detailnú anatómiu ľudského tela od kostí a svalov až po nervovú sústavu. Nejde však len o teoretický výklad, ale aj praktickú ukážku priamo na simulácii ľudského tela. Vďaka špeciálnemu režimu je študentovi umožnené nahľadnuť aj do jeho najmenších zákutí. Zároveň ponúka možnosť otestovania vedomostí pomocou testov a kvízov, ktoré sú po dokončení automaticky vyhodnotené.

\begin{figure}[tbh]
	\centering
	\includegraphics[scale=0.5]{Img-3.png}
	\caption{"Human Anatomy" (Zdroj:\cite{Img-Antom})}
	\label{obr-3}
\end{figure}

\section{Záver}

Proces vzdelávania prechádza vývojom od nepamäti, jeden z najväčších zlomov avšak nastal až v dvadsiatom prvom storočí, kedy sa postupne začal prepájať s digitálnym priestorom. Ten momentálne slúži primárne na uskladnenie vedomostí, čo by sa v blízkej budúcnosti mohlo zmeniť. Súčasťou tohto priestoru sú totiž aj hry, ktoré bolo v minulosti nemysliteľné spájať so vzdelávaním. To sa však postupom času zmenilo a v súčasnosti predstavujú novú príležitosť ako zlepšiť a zefektívniť vyučovací proces. V našom článku sme poukázali na výhody použitia hier v procese výučby. Táto nová metóda samozrejme prináša aj svoje nevýhody, tie sú ale oproti výhodám v menšine. Samotná implementácia hier do vyučovacieho procesu je náročná a zdĺhavá, ale vo výsledku prinesie osoh vyučujúcim aj študentom.

\bibliography{literatura}
\bibliographystyle{plain}
\end{document}

\section{Tvorba článku a reakcie na témy z prednášok}

\begin{figure}[tbh]
	\centering
	\caption{Diagram - Proces tvorby článku (A. Rybár, 2022)}
	\includegraphics[scale=0.35]{Diagram-1.png}
	\label{obr-4}
\end{figure}

 \paragraph{Inžinierska práca v informatike a písanie technického textu}
V rámci tejto prednášky sme sa dozvedeli, čím sa zaoberá inžinier a aké vlastnosti by mal mať. Jej súčasťou bolo aj základné oboznámenie sa s nástrojom Git. Osobne mi prišla zaujímavejšia a užitčnejšia druhá časť prednášky, ktorá sa týkala Gitu a práce s ním. Preto by som si vedel predstaviť skrátenie prvej časti, ktorá sa týkala vlastností inžiniera a naopak predĺženie časti, ktorá pojednávala o Gite a jeho využití. Celkovo však na mňa prednáška pôsobila pozitívne.

 \paragraph{Grafické vyjadrenie informácií v informatike}
Táto prednáška pojednávala o grafickom vyjadrení informácií. Najskôr sme boli teoreticky oboznámení s definíciou vizualizácie a načo nám vlastne slúži. Ďalšiu časť tvorili kognitívne mapy a ich všeobecné využitie. Na záver prednášky sme sa zaoberali rôznymi grafickými editormi, ktorými sa dajú vyjadriť rôzne typy diagramov. Osobne mi prišla najzaujímavejšia posledná časť, teda oboznámenie sa z rôznymi grafickými editormi a taktiež typmi diagramov. Na túto poslednú časť prednášky by som si vedel predstaviť pokračovanie, poprípade rozšírenie, ktoré by bolo zamerané na hlbšie oboznámenia sa z rôznymi typmi diagramov a grafických editorov. Celkovo však mala pre mňa prednáška pozitívny prínos.

 \paragraph{Načo budem inžinierom?(bakalárom)}
Na tejto prednáške sme boli oboznámení v čom spočíva byť inžinierom (bakalárom). Zároveň nám boli predstavené možné budúce povolania, o ktoré je možné sa uchádzať po absolvovaní štúdia, a prekážky, ktoré musíme prekonať nato, aby sme toto štúdium úspešne absolvovali. Osobne mi prišla táto prednáška s hosťom prínosná a zaujímavá. Zároveň ma naviedla k zamysleniu sa nad budúcim povolaním. Preto na mňa celkovo pôsobila veľmi pozitívne.

\end{document}