\documentclass[10pt,twoside,slovak,a4paper]{article}

\usepackage[slovak]{babel}
\usepackage[IL2]{fontenc}
\usepackage[utf8]{inputenc}
\usepackage{graphicx}
\usepackage{url}
\usepackage{hyperref}
\usepackage{cite}

\pagestyle{headings}

\title{Hry ako jeden z nástrojov vyučovacieho procesu\thanks{Semestrálny projekt v predmete Metódy inžinierskej práce, ak. rok 2022/23, vedenie: Meno Priezvisko}}

\author{Meno Priezvisko\\[2pt]
	{\small Slovenská technická univerzita v Bratislave}\\
	{\small Fakulta informatiky a informačných technológií}\\
	{\small \texttt{...@stuba.sk}}
	}

\date{\small 6. november 2022}



\begin{document}

\maketitle

\begin{abstract}

Moderné technológie, ktoré sú súčasťou nášho každodenného života, ponúkajú široké možnosti využitia v rôznych odboroch a sférach. Jednou z nich je aj použitie hier v procese výučby, v ktorom v súčasnosti nezohrávajú veľkú rolu, napriek svojmu potenciálu. Vo našom článku sa budeme venovať práve tomuto potenciálu, teda možnostiam využitia hier vo vzdelávacom procese, čo by pomohlo zjednodušiť a zlepšiť celkovú výučbu. Zameriame sa na porovnanie výučby klasickej a výučby s použitím vzdelávacích hier. Zároveň poukážeme na výhody plynúce z využitia hier v samotnom procese výučby v jednotlivých stupňoch vzdelávania. Pre každý stupeň vyberieem konkrétny príklad, na ktorom vysvetlíme prínos herných prvkov do vyučovania a uvedieme akým spôsobom rozvíjajú zručnosti a schopnosti študenta.

\end{abstract}



\section{Úvod}

Dnešná doba je plná technológií, ktoré nám poskytujú množstvo možností vo všetkých sférach nášho každodenného života. Výnimkou nie je ani proces vyučovania, v ktorom nám moderné technológie, konkrétne hry\cite{Zea2009-eh}, môžu pomôcť. Tie nám ponúkajú veľké množstvo príležitostí, ktoré priam vyzývajú k tomu, aby sa využívali a doplnili tak alebo úplne nahradili klasické vyučovacie metódy. Najmä v posledných rokoch začínajú vychádzať na povrch nedostatky klasického vyučovania. Svojou monotónnosťou, pasivitou a zložitosťou pôsobia na študentov demotivujúco a pomaly, ale isto im uberajú z chuti vzdelávať sa. Tieto nedostatky by mohlo napomôcť vyriešiť práve využitie hier v procese vyučovania, ktorému sa budeme venovať v našom článku. 

\section{Výhody použitia hier vo výučbe} \label{vyhody}

Videohry ako také sa v dnešnej dobe stali novou voľnočasovou aktivitou detí. Už od najmenšieho veku sa deti oboznamujú s videohrami, kde si môžu rozvíjať motoriku, reflexy, trpezlivosť, vytrvalosť a ďalšie schopnosti\cite{Chen2012-ao}. Digitálne hry, ak sú kvalitne vytvorené, dokážu upútať pozornosť študentov viac, ako klasické vyučovacie prostriedky. Digitálne hry, ak sú kvalitne vytvorené, dokážu upútať pozornosť študentov viac, ako klasické vyučovacie prostriedky. V súčasnosti už prebehli a zároveň prebiehajú rôzne výskumy, ktoré skúmajú čo prinášajú tieto metódy priamo v aktívnom procese výučby. Je známe, že výučba hrou je omnoho prijateľnejšia pre väčšinu študentov ako výučba klasická. Aj pri menej obľúbenom predmete má vyučujúci prostredníctvom použitia hry viac možností zatraktívniť tento predmet a zaujať tak viac poslucháčov, urobiť z tohto predmetu zábavu a nenásilným spôsobom priviesť študentov k vedomostiam bez toho, že by si to primárne uvedomovali. Tento spôsob výučby má mnoho kladov ako pre poslucháčov, tak i pre vyučujúcich.

\subsection{Prínosy pre študentov} \label{vyhody:studenti}

Zásadnou výhodou takejto metódy výučby je možnosť aktívneho zapojenia študentov do vyučovacieho procesu. Je všeobecne známe, že ak si študent preberanú látku prakticky vyskúša, upevní si tak získané vedomosti. Avšak pri klasickej metóde vyúčby bolo takmer jedinou možnosťou aktívneho zapojenia sa do vyučovania skúšanie pri tabuli, ktoré je pre mnohých študentov stresujúce. Využitím hier by sa tento problém vyriešil, pretože pri samotnom hraní by si študent upevnil alebo doplnil svoje vedomosti.

\subsection{Prínosy pre vyučúcich} \label{vyhody:vyucujuci}

Využitie hier vo vyučovaní zväčšuje záujem o predmet zo strany študentov\cite{Ucenie}, čo učiteľovi zľahčuje celý vyučovací proces, pretože nemusí venovať toľko usília, aby zaujal študentov a môže sa tak naplno venovať preberanej látke. Za prínos, ktorý zľahčuje prácu vyučujúcemu, môžeme považovať aj automatickú kontrolu a vyhodnotenie výsledkov prejdenej časti hry. S tým súvisí aj ďalšia výhoda, ktorou je rýchla spätná väzba. Tú učiteľ obdrží hneď po ukončení segmentu hry a tak môže prakticky okamžite začať s jej analýzou. Takto môže veľmi rýchlo spoznať silné a slabé stránky študentov, na základe ktorých môže následne prispôsobiť metódy výučby a tak efektívnejšie dosahovať stanovené ciele vzdelávania.

\section{Nevýhody použitia hier vo výučbe} \label{nevyhody}

Pri zavádzaní hier do systému výučby sa však stretávame aj z mnohými nevýhodami. Najčastejšie je to v technickom riešení na školách, chýbajú moderné a výkonné počítače, ktoré by spĺňali požadované parametre. Pre vyučujúcich sú dané určité oklieštené osnovy pre vzdelávací proces, čo vytvára veľmi malý priestor na zavádzanie a použitie nových systémov do procesu vyučovania.  Medzi nevýhody môžeme zaradiť aj nevôľu zo strany vyučujúcich zavádzať nové systémy v rámci výučby. Môže to byť spôsobené nezáujmom vyučujúceho o hry ako také alebo je problém vek a s ním spojené vnímanie zmien a prístup k moderným technológiám.

\section{Kritéria pri výbere hier} \label{kriteria}

Pri výbere vhodných hier k výučbe je treba, aby vyučujúci prihliadal na určité kritéria, ktoré by mali vybrané hry spĺňať. V prvom rade je dôležité, aby hra študentom poskytovala hodnotné informácie, ktoré im uľahčia pochopiť obsah daného učiva. Umožnila im rozširovať si vedomosti a obzory v rámci vyučovacieho procesu. A v neposlednom rade zaviedla do výučby nové a moderné techniky, ktoré by pomáhali rozvíjať schopnosti jednotlivca. Hra by nemala byť časovo náročná, musí byť použiteľná v praxi a mala by byť vhodná pre určenú skupinu študentov na základe úrovne dosiahnutých vedomostí.

\section{Využitie hier vo výučbe} \label{vyuzitie}

\begin{figure}[tbh]
	\centering
	\includegraphics[scale=0.3]{Img-2.png}
	\caption{"Civilization" (Zdroj:\cite{Img-Civil})}
	\label{obr-2}
\end{figure}

\subsection{Civilization} \label{hra-2}


\section{Záver}

\bibliography{literatura}
\bibliographystyle{plain}
\end{document}
